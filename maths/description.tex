\documentclass[a4paper,11pt,draft]{article}

\usepackage[frenchb]{babel}
\usepackage[utf8]{inputenc}
\usepackage[T1]{fontenc}

\usepackage{amsmath}
\usepackage{amsthm}
\usepackage{amstext}
\usepackage{amsfonts}
\usepackage{amssymb}
\usepackage{dsfont}
\usepackage{stmaryrd}

\usepackage{a4wide}
\usepackage{graphicx}
\usepackage{multimedia}
%\usepackage{hyperref}

\usepackage[notref]{showkeys}

\def\R{\mathbb{R}}
\def\N{\mathbb{N}}
\def\C{\mathbb{C}}
\def\Z{\mathbb{Z}}
\def\Q{\mathbb{Q}}

\def\Ac{\mathcal{A}}
\def\Bc{\mathcal{B}}
\def\Cc{\mathcal{C}}
\def\Dc{\mathcal{D}}
\def\Ec{\mathcal{E}}
\def\Fc{\mathcal{F}}
\def\Gc{\mathcal{G}}
\def\Hc{\mathcal{H}}
\def\Ic{\mathcal{I}}
\def\Jc{\mathcal{J}}
\def\Kc{\mathcal{K}}
\def\Lc{\mathcal{L}}
\def\Mc{\mathcal{M}}
\def\Nc{\mathcal{N}}
\def\Oc{\mathcal{O}}
\def\Pc{\mathcal{P}}
\def\Qc{\mathcal{Q}}
\def\Rc{\mathcal{R}}
\def\Sc{\mathcal{S}}
\def\Tc{\mathcal{T}}
\def\Uc{\mathcal{U}}
\def\Vc{\mathcal{V}}
\def\Wc{\mathcal{W}}
\def\Xc{\mathcal{X}}
\def\Yc{\mathcal{Y}}
\def\Zc{\mathcal{Z}}

\def\Id{\mathrm{Id}}

\newtheorem{exercice}{Exercice}
\newtheorem{theorem}{Theorem}
\newtheorem{definition}{Definition}
\newtheorem{proposition}{Proposition}
\newtheorem{corollary}{Corollary}
\newtheorem{lemma}{Lemma}
\newtheorem{remark}{Remark}
\newtheorem{notation}{Notation}

\title{Contrôle optimal pour Lotka-Volterra en mode bistable. Notations standardisées}

\author{}
\date{}

\begin{document}

\maketitle

\section{Problèmatique}
\label{sec:1}

On s'intéresse au système suivant
\begin{equation} \label{eq:1}
    \begin{cases}
        \dot{x}=x(a(\bar{x}-x)-by)\\
        \dot{y}=y(d(\bar{y}-y)-cx)
    \end{cases}
\end{equation}
les coefficients $a,b,c,d,\bar{x}, \bar{y}$ sont tous positifs.
\
On contrôle ce système à l'aide d'un contrôle scalaire multiplicatif $u(t)\in [0,M]$. Le système contrôlé s'écrit :
\begin{equation} \label{eq:2}
    \begin{cases}
        \dot{x}=x(a(\bar{x}-x)-by-\alpha u)\\
        \dot{y}=y(d(\bar{y}-y)-cx-\beta u)
    \end{cases}
\end{equation}
On appelle {\it système saturé}\ le système \eqref{eq:2} avec $u\equiv M$ (cf. \eqref{eq:3}).

Notons que le contrôle multiplicatif assure la conservation du quadrant positif pour \eqref{eq:2}.
{\bf But : } On cherche à utiliser le contrôle $u$ pour conduire l'espèce $x$ à l'extinction
alors qu'elle survivrait sans le contrôle.

La première idée est d'utiliser des techniques issues du contrôle optimal. Plus précisément on
cherche à minimiser la fonctionnelle
\begin{equation}
    J(u):=\frac{x^2(T) + (y(T)-\bar{y})^2}{2}.
\end{equation}
Le principe du maximum de Pontryagin permet de constater que le contrôle optimal est forcément
Bang-Bang ce qui conduit à analyser les systèmes autonomes pour $u=0$ et $u=M$.

\section{Le système sans contrôle}
\subsection{Les solutions stationnaires}
Les isoclines du systèmes \eqref{eq:1} sont simples et s'écrivent :
$$\dot{x}=0, \quad x=0\text{ ou } ax+by=a\bar{x}.$$
$$\dot{y}=0, \quad y=0\text{ ou } cx+dy=d\bar{y}.$$

$O=(0,0)$ est toujours un état stationnaire. On a également toujours les deux solutions stationnaires semi-triviales :
$\bar{X}=(\bar{x},0)$ et $\bar{Y}=(0,\bar{y})$.
Enfin, supposons que $ad-bc\neq 0$. Il y a éventuellement une solution stationnaires positives :

$$\bar{C}=(x_c,y_c):=\left(\frac{d(a\bar{x}-b\bar{y})}{ad-bc},\frac{a(d\bar{y}-c\bar{x})}{ad-bc}\right).$$

Cette solution existe et est positive si et seulement chacune de ses composantes est positive.  
\subsection{Stabilité linaire}
\'Etudions les Jacobiennes en tout point. On note $J_A$ la jacobienne du système \eqref{eq:1} en un point $A=(x_A,y_A)$.\\
On a 
\begin{equation*}
J_O=
\begin{pmatrix}
    a\bar{x}&0\\
    0&d\bar{y}
\end{pmatrix}.
\end{equation*}

{\bf $O$ est donc toujours instable.}

\begin{equation*}
J_{\bar{X}}=
\begin{pmatrix}
    -a\bar{x}&-b\bar{x}\\
    0&d\bar{y}-c\bar{x}
\end{pmatrix}.
\end{equation*}
{\bf $\bar{X}$ est donc (asymptotiquement) stable si et seulement si $d\bar{y}-c\bar{x} <0$ }

$$J_{\bar{Y}}=
\begin{pmatrix}
    a\bar{x}-b\bar{y}&0\\
    -d\bar{y}&-c\bar{x}
\end{pmatrix}.
$$
{\bf $\bar{Y}$ est donc (asymptotiquement) stable si et seulement si $a\bar{x}-b\bar{y} <0$ }


$$J_{\bar{C}}=
\begin{pmatrix}
    -ax_c&-dx_c\\
    -c y_c&-dy_c
\end{pmatrix}.
$$
{\bf Sa trace étant négative, $\bar{C}$ est  (asymptotiquement) stable si et seulement si son déterminant est positif à savoir $ad-bc >0$ et est instable si $ad-bc<0$ }.


\subsection{Différents régimes}
Si $ad-bc\neq 0$ il y a quatre régimes possibles.

\paragraph{Bistable :}
Si $d\bar{y}-c\bar{x} <0$ et $a\bar{x}-b\bar{y} < 0$ alors $x_c>0$, $y_c>0$ et  $ad-bc<0$. Ce régime est le régime bistable :
$\bar{X}$ est stable, $\bar{Y}$ est stable et par conséquent $\bar{C}$ est instable.

\paragraph{Coexistence :}
Si $d\bar{y}-c\bar{x} >0$ et $a\bar{x}-b\bar{y} > 0$ alors $x_c>0$, $y_c>0$ et  $ad-bc>0$. Ce régime est le régime de coexistence :
$\bar{X}$ est instable, $\bar{Y}$ est instable et par conséquent $\bar{C}$ est stable.

\paragraph{Exclusion de $y$ :}
Si $d\bar{y}-c\bar{x} <0$ et $a\bar{x}-b\bar{y} > 0$ alors $\bar{C}$ n'existe pas. Ce régime est un régime d'exclusion compétitive (de $y$):
$\bar{X}$ est stable, $\bar{Y}$ est instable.

\paragraph{Exclusion de $x$ :}
Si $d\bar{y}-c\bar{x} >0$ et $a\bar{x}-b\bar{y} < 0$ alors $\bar{C}$ n'existe pas. Ce régime est  un régime d'exclusion compétitive (de $x$): :
$\bar{X}$ est instable, $\bar{Y}$ est stable.
%%%%%%%%%%%%%%%%%%%%%%%%%%%%%%%%%%%%%%%%%%%%%%%%%%%%%%%%%%%%%%%%%%%%%%%%%%%%%%%%%%%%%%%%%%%%%%%%%%%%%%%%%%%%%%%%%%
%%%%%%%%%%%%%%%%%%%%%%%%%%%%%%%%%%%%%%%%%%%%%%%%%%%%%%%%%%%%%%%%%%%%%%%%%%%%%%%%%%%%%%%%%%%%%%%%%%%%%%%%%%%%%%%%%%
%%%%%%%%%%%%%%%%%%%%%%%%%%%%%%%%%%%%%%%%%%%%%%%%%%%%%%%%%%%%%%%%%%%%%%%%%%%%%%%%%%%%%%%%%%%%%%%%%%%%%%%%%%%%%%%%%%
\section{Le système avec contrôle saturé}
Dans le cas d'un contrôle saturé $u(t)=M$, le système \eqref{eq:2} se réécrit sous la forme du système \eqref{eq:1} en posant 
$$\bar{x}_M=\bar{x}-\frac{\alpha M}{a},\quad \bar{y}_M=\bar{y}-\frac{\beta M}{d}.$$

\begin{equation} \label{eq:3}
    \begin{cases}
        \dot{x}=x(a(\bar{x}_M-x)-by)\\
        \dot{y}=y(d(\bar{y}_M-y)-cx)
    \end{cases}
\end{equation}
Attention cependant, ce système a la même structure que \eqref{eq:1} si $\bar{x}_M$ et $\bar{y}_M$ sont positive (cf hypothèse $(C_0)$ ci dessous).
\subsection{Les solutions stationnaires}
Les isoclines du systèmes \eqref{eq:3} sont simples et s'écrivent :
$$\dot{x}=0, \quad x=0\text{ où } ax+by=a\bar{x}_M.$$
$$\dot{y}=0, \quad y=0\text{ où } cx+dy=d\bar{y}_M.$$

$O=(0,0)$ est toujours un état stationnaire.


 On a également  les deux solutions stationnaires semi-triviales :
$\bar{X}_M=(\bar{x}_M,0)$ et $\bar{Y}_M=(0,\bar{y})$.\\
{\bf Ces solutions sont positives si et seulement si 
$$(C_0)\;:\;\begin{cases} \bar{x}_M=\bar{x}-\frac{\alpha M}{a}>0 \\ \bar{y}_M=\bar{y}-\frac{\beta M}{d}>0.\end{cases}$$}

Comme dans le cas précédent, il y a  y a éventuellement une solution stationnaires positives :

$$\bar{C}_M=(x_{c,M},y_{c,M})=\left(\frac{d(a\bar{x}_M-b\bar{y}_M)}{ad-bc},\frac{a(d\bar{y}_M-c\bar{x}_M)}{ad-bc}\right).$$

Cette solution existe et est positive si et seulement chacune de ses composantes est positive.  
\subsection{Stabilité linaire du système saturé}

L'étude précédente nous donne les conditions suivantes :

{\bf $O$ est  instable si et seulement si $\bar{x}_M$ ou $\bar{y}_M$ est positif.}

{\bf Si $\bar{X}_M$ existe, il est  (asymptotiquement) stable si et seulement si $d\bar{y}_M-c\bar{x}_M <0$.}

{\bf Si $\bar{Y}_M$ existe, il est (asymptotiquement) stable si et seulement si $d\bar{y}-c\bar{x} <0$.}

{\bf Si $P_M$ existe, il est $\begin{cases} \text{(asymptotiquement) stable si $ad-bc>0$}\\
\text{instable si $ad-bc<0$ }\end{cases}$.}

\subsection{Différents régimes}
Si $ad-bc\neq 0$ il y a quatre régimes possibles pour le système saturé :

\paragraph{Bistable :}
Si $d\bar{y}_M-c\bar{x}_M <0$ et $a\bar{x}_M-b\bar{y}_M < 0$ alors $x_{c,M}>0$, $y_{c,M}>0$ et  $ad-bc<0$. Ce régime est le régime bistable :
$\bar{X}_M$ est stable, $\bar{Y}_M$ est stable et par conséquent $\bar{C}_M$ est instable.

\paragraph{Coexistence :}
Si $d\bar{y}_{M}-c\bar{x}_{M} >0$ et $a\bar{x}_{M}-b\bar{y}_{M} > 0$ alors $x_{c,M}>0$, $y_{c,M}>0$ et  $ad-bc>0$. Ce régime est le régime de coexistence :
$\bar{X}_{M}$ est instable, $\bar{Y}_{M}$ est instable et par conséquent $\bar{C}_{M}$ est stable.

\paragraph{Exclusion de $y$ :}
Si $d\bar{y}_{M}-c\bar{x}_{M} <0$ et $a\bar{x}_{M}-b\bar{y}_{M} > 0$ alors $\bar{C}_{M}$ n'existe pas. Ce régime est un régime d'exclusion compétitive (de $y$):
$\bar{X}_M$ est stable, $\bar{Y}_M$ est instable.

\paragraph{Exclusion de $x$ :}
Si $d\bar{y}_M-c\bar{x}_M >0$ et $a\bar{x}_M-b\bar{y}_M < 0$ alors $\bar{C}_{M}$ n'existe pas. Ce régime est  un régime d'exclusion compétitive (de $x$): :
$\bar{X}_M$ est instable, $\bar{Y}_M$ est stable.

\section{Méthodes numériques}

Bien qu'on puisse utiliser les équations fournies par le principe du Maximum de Pontryagin, la
résolution du problème est délicate car on a des conditions aux bords ($t=0$ et $t=T$) plutôt
qu'un simple problème de Cauchy et la sensibilité aux conditions initiales est très grande pour
faire marcher une méthode de Tir. On s'oriente donc plutôt vers une résolution de l'équation
d'Hamilton-Jacobi associée à la fonction valeur.

On introduit donc 
\begin{equation}
    V(t_0, x_0, y_0) := \underset{u\in L^\infty([t_0,T]; [0,M])}{\inf} \frac{x^2(T) +
    (y(T)-\bar{y})^2}{2}. 
\end{equation}
avec $x,y$ solution de \eqref{eq:2} satisfaisant $x(t_0)=x_0$ et $y(t_0)=y_0$.

Le principe de programmation dynamique permet alors d'écrire que $V$ satisfait
\begin{equation}\label{eq:4}
    V(t_0,x_0,y_0) = \underset{u\in L^\infty([t_0,t_0+h],[0,M])}{\inf} V(t_0+h, x(t_0+h), y(t_0+h))
\end{equation}
pour $h$ positif petit.  On pourrait alors montrer que $V$ est solution au sens des solutions
de viscosité de l'équation
\begin{equation}
    \begin{cases}
        \partial_t V + H(x,y,\partial_x V, \partial_y V)=0,\\
        V(T,x,y)= \frac{x^2 + (y-\bar{y})^2}{2}.
    \end{cases}
\end{equation}
où $H$ est donnée par
\begin{align}
    H(x,y,p,q) & = \underset{u\in[0,M]}{\inf} p x (a (\bar{x} - x) - bx -\alpha u)
    + q y (d (\bar{y} - y) -c x - \beta u)\\
               & = p x (a (\bar{x} - x) - bx) + q y (d (\bar{y} - y) -c x ) - M (\alpha p x +
               \beta q y)^+.
\end{align}
En fait d'un point de vue numérique on repart de \eqref{eq:4} que l'on discrétise pour obtenir
\begin{equation}\label{eq:5}
    V(t, x, y) = \underset{u\in[0,M]}{\inf} V(t+\Delta t, x + \Delta t f(x,y,u), y + \Delta t
    g(x, y,u)).
\end{equation}
avec $f$ et $g$ donnés par
\begin{equation}
    \begin{cases}
        f(x,y,u) = x (a (\bar{x} - x) - b y - \alpha u)\\
        g(x,y,u) = y (d (\bar{y} - y) - c x - \beta u).
    \end{cases}
\end{equation}
On échantillonne alors $V$ sur une grille rectangulaire de $[0,\bar{x}] \times [0, \bar{y}]$
(ce domaine étant invariant par les flux quand $u$ est positif). On évalue alors le terme de
droite dans \eqref{eq:5} via une interpolation adaptée à une grille rectangulaire (plus simple
et précis que les élements finis P1).

On a au passage une condition CFL qui nous dit que en un pas de temps le flux ne doit pas nous
faire sortir des cases adjacentes à notre point d'échantillonnement.
\end{document}
